\chapter{Záver}
\label{kapitola7}
Cieľom tejto bakalárskej práce bolo vytvoriť webovú aplikáciu pre správu prezentácií s obsahom v značkovacom jazyku Markdown. Výsledná aplikáciu umožní užívateľom vytvárať, upravovať a odstraňovať jednotlivé prezentácie. Veľkou výhodou je možnosť verzovania prezentácií. Aplikácia poskytuje užívateľom možnosť uloženia úprav v prezentácii pod novou verziou a uchovanie starých verzií. Jednotlivé verzie sú voľne dostupné pre užívateľa, ktorý si ich môže kedykoľvek zobraziť, upraviť, stiahnuť, alebo odstrániť. Užívateľ má naďalej možnosť nahliadnuť do stránok verzie hneď na domovskej stránke, čo uľahčí voľbu tej správnej verzie pred jej zobrazením, alebo úpravou. 

Editor obsahuje užitočné nástroje pre tvorbu Markdown obsahu s okamžitým náhľadom na sformátovanú stránku. Aplikácia umožňuje prehľadnú orientáciu medzi stránkami pomocou bočného panela a režimu mriežkového pohľadu. Postupnosť jednotlivých stránok sa dá jednoducho upraviť pomocou technológie ťahaj-a-pusti. Aplikácia naďalej implementuje podporu pre kopírovanie stránok medzi prezentáciami a ich verziami.

Serverová aj klientská časť aplikácie bola implementovaná v jazyku TypeScript. Voľbu TypeScriptu považujem za veľmi pozitívnu. Keďže jazyk som pomerne dobre ovládal už pred začatím implementácie projektu, jej použitie neprinieslo žiadne ťažkosti pri programovaní v nej, ale práve naopak. Jednotlivé časti aplikácie neboli hneď zdokumentované pri ich implementácii a po mesiacoch vrátenia k nim, zdrojový kód bol stále prehľadný a logika za ňou jednoducho pochopiteľná kvôli vlastnostiam TypeScriptu. 

Klientská časť aplikácie bola implementovaná pomocou JavaScriptového frameworku Vue.js, s pomocou nadstavby Nuxt.js. Pri implementácii sa využila technológia Composition API, ktorá je dostupná vo Vue.js verzii 3. Po úspešnej implementácii projektu viem posúdiť, že tvorba komponentov a znovupoužiteľnej logiky cez Comosition API priniesli veľa výhod pri implementácii. Umožnila vyčlenenie znovupoužiteľnej logiky a jednotlivých častí zdrojového kódu, ktoré sa označujú ako boilerplate kódy a zaberajú veľa miesta, do zvlášť súborov. Výsledkom je prehľadnejšia orientácie v komponentoch aplikácie. 

Pri implementácii sa použil verzovací systém Git a nástroj GitHub, ktorý pomohol v organizácii úloh projektu. Pri pridaní nových zmien do projektu sa testovala celková funkčnosť aplikácie pomocou nástroja Cypress, aby sa zaručilo očakávané správanie aplikácie.

Možným rozlíšením aplikácie by mohla byť funkcionalita porovnania dvoch verzií prezentácie, pričom by sa zvolili dve verzie a aplikácia by zobrazila rozdiely v ich obsahu. Porovnanie by naďalej zobrazilo vykonané akcie v novšej verzii, ako napríklad pridanie a odstránenie strán.