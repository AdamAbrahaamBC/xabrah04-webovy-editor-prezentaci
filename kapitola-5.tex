\chapter{Implementácia}

\section{Git a GitHub}
Pred začatím písania zdrojového kódu sa najprv spojazdnil verzovací systém Git cez GitHub. Cieľom bolo verzovanie samotného zdrojového kódu aj dokumentácie. Pre jednoduchšiu orientáciu v repozitároch sa vytvorila GitHub organizácia, ktorá obsahuje monorepozitár pre zdrojový kód a repozitár pre dokumentáciu na jednom mieste. Monorepozitár je repozitár obsahujúci viacero projektov, v tomto prípade projekt klienta a servera. Organizácia aj repozitáre sú verejne dostupné pod odkazom:
    \begin{verbatim}
        https://github.com/AdamAbrahaamBC
    \end{verbatim}

GitHub ponúka projektom organizačnú tabuľku. Tabuľka pomáha v organizovaní úloh a poskytuje jednoduchší prehľad projektu. Môže mať ľubovolný počet kolóniek do ktorých sa prideľujú jednotlivé úlohy. V tomto projekte bolo postačujúce vytvorenie troch kolóniek. Kolónka \texttt{To Do} obsahuje ešte nezačaté úlohy, kolónka \texttt{In progress} úlohy, ktoré sa práve riešia a kolónka \texttt{Done} už dokončené a zatvorené úlohy. Pri vytvorení novej úlohy sa úloha automaticky priradí do kolónky \texttt{To Do}. Pri zvolení následujúcej úlohy je manuálne premiestnená do kolónky \texttt{In Progress} a pri jej zatvorení sa automaticky premiestni do kolónky \texttt{Done}.


K jednotlivým úlohám sú pridelené štítky. Štítky označujú o aký typ úlohy sa jedná. Dostupné štítky projektu sú následovné:

    \begin{itemize}
        \item\textbf{backend} - označuje úlohy týkajúce sa backendu
        \item\textbf{frontend} - označuje úlohy týkajúce sa frontendu
        \item\textbf{db} - označuje úlohy na databáze
        \item\textbf{bug} - označuje úlohy obsahujúce chybu v aplikácii
        \item\textbf{todo} - označuje zatiaľ nezačaté úlohy
        \item\textbf{done} - označuje dokončené úlohy
    \end{itemize}